\documentclass[svgnames]{report}
\usepackage{wordle}
\usepackage{manfnt}
\usepackage{enumitem}
\setlist[description]{font=\sffamily\bfseries\color{ForestGreen},labelwidth=\textwidth}
\usepackage{booktabs}
\usepackage[a4paper,margin=18mm]{geometry}
\synctex=1
\parindent=0pt
\parskip=2mm

\usepackage{cmap} % fix search and cut-and-paste in Acrobat
\setcounter{secnumdepth}{0}

\usepackage{listings}\lstset{language=[LaTeX]TeX,framerule=2pt}
\lstset{language=[LaTeX]TeX,
        texcsstyle=*\bfseries\color{FireBrick},
        backgroundcolor=\color{Ivory},
        numbers=none,
        breaklines=true,
        keywordstyle=\color{FireBrick},
        commentstyle=\color{ForestGreen},
        tabsize=2,
        morekeywords={Wordle,wordle},
        resetmargins=true,
}
% hyperref links to ctan
\newcommand\ctan[1]{\href{https://www.ctan.org/pkg/#1}{\texttt{#1}}}

\newcommand\Section[1]{\subsection{\textcolor{DarkGreen}{#1}}}

\NewDocumentCommand\option{ mo }{%
  \medskip\textbf{\textcolor{DarkGreen}{#1}}%
  \IfNoValueF{#2}{ (default: \textcolor{SeaGreen}{#2})}%
  \index{#1}%
}

%%%%%%%%%%%%%%%%%%%%%%%%%%%%%%%%%%%%%%%%%%%%%%%%%%%%%%%%%%%%%%%%%%%%%%%%%%%%%%%%%
\usepackage{imakeidx}
\indexsetup{level=\section*, toclevel=section, noclearpage}
\makeindex[intoc,columns=3]

%%%%%%%%%%%%%%%%%%%%%%%%%%%%%%%%%%%%%%%%%%%%%%%%%%%%%%%%%%%%%%%%%%%%%%%%%%%%%%%%%
\usepackage[skins,listings]{tcolorbox}

\DeclareTotalTCBox\keyword{ O{} v }{
  fontupper=\sffamily,
  nobeforeafter,
  skin=tile,
  verbatim,
  on line,
  tcbox raise base,
  top=0pt,bottom=0pt,left=0mm,right=0mm,
  colback=OldLace,
  colupper=ForestGreen,
  #1}
{#2}

\lstdefinestyle{tikz}{style=tcblatex,
  classoffset=0,
  texcsstyle=*\color{DarkGoldenrod},%
  deletetexcs={begin, end},
  moretexcs={,%
    node,
    draw,
    true,
    false
  },%
  classoffset=1,
  keywordstyle=\color{ForestGreen},%
  morekeywords={wordle},
  classoffset=2,
  keywordstyle=\color{Crimson},%
  morekeywords={
    border,
    case, sensitive,
    correct,
    font,
    lower,
    name,
    natural
    placed,
    rows,
    separation,
    shape,
    size,
    strict,
    text,
    tikz,
    unmarked,
    upper,
    wrong,
  }
}

\DeclareTCBListing{example}{ !O{} }{%
  skin=bicolor,
  colframe=MediumSeaGreen,
  colbacklower=OldLace,
  colback=LawnGreen!20,
  lefthand width=50mm,
  listing style=tikz,
  sidebyside,
  sidebyside align=center,
  sidebyside gap=4mm,
  text and listing,
  text outside listing,
  boxsep = 0pt,
  #1
}

%%%%%%%%%%%%%%%%%%%%%%%%%%%%%%%%%%%%%%%%%%%%%%%%%%%%%%%%%%%%%%%%%%%%%%%%%%%%%%%%%


\makeatletter
\author{Andrew Mathas}
\usepackage{tikz}
\usetikzlibrary{shadows.blur}
\tikzset{shadowed/.style={blur shadow={shadow blur steps=5},
                          bottom color=ForestGreen!30,
                          draw=MediumSeaGreen!70,
                          shade,
                          font=\normalfont\Huge\bfseries\scshape,
                          rounded corners=8pt,
                          top color=SeaGreen,
      },
      boxes/.style={draw=SeaGreen,
                    fill=Cornsilk,
                    font=\sffamily\small,
                    inner sep=5pt,
                    rectangle,
                    rounded corners=8pt,
                    text=DarkGreen,
     }
}
\newcommand\Wordle{%
  \begin{tikzpicture}[remember picture,overlay]
      \node[yshift=-3cm] at (current page.north west)
        {\begin{tikzpicture}[remember picture, overlay]
          \draw[shadowed](30mm,0) rectangle node[white]{Wordle} (\paperwidth-30mm,16mm);
          \node[anchor=west,boxes] at (4cm,0cm) {\@author};
          \node[anchor=east,boxes] at (\paperwidth-4cm,0) {\wordle@version};
         \end{tikzpicture}
        };
   \end{tikzpicture}
   \vspace*{20mm}
}

\def\@oddfoot{\textsc{Wordle} -- \wordle@version\hfill\thepage}

\usepackage[colorlinks=true,linkcolor=blue,urlcolor=MediumBlue]{hyperref}
\hypersetup{
  pdfcreator={ Generated by pdfLaTeX },
  pdfinfo={
    Author  ={ Andrew Mathas },
    Keywords={ wordle },
    License ={ LaTeX Project Public License v1.3c or later },
    Subject ={ LaTeXing Wordle puzzles },
    Title   ={ Wordle - \wordle@version }
  },
}
\makeatother

\begin{document}

  \Wordle

  This package provides a stupidly feature-rich way of creating
  \keyword{Wordle} puzzles in \LaTeX, which
  \href{https://en.wikipedia.org/wiki/Wordle}{Wikipedia} describes as a
  hugely popular web-based word game created and developed by Welsh
  software engineer Josh Wardle. This package is an enhanced version of
  my solution to a question asked by \textcolor{FireBrick}{valerio\_new} on
  \href{https://tex.stackexchange.com/questions/659860/wordle-like-colored-letter-boxes-in-latex/660056#660056}{tex.stackexchange.com}.


  This package provides the \keyword{wordle} environment, which uses the
  syntax:

  \begin{tcblisting}{
    listing only,
    skin=bicolor,
    colframe=MediumSeaGreen,
    colbacklower=OldLace,
    colback=LawnGreen!20,
    listing style=tikz,
    boxsep = 0pt,
  }
        \begin{wordle}[wordle options]{solution}[optional tikz commands]
          guess1 guess2 ...
        \end{wordle}
  \end{tcblisting}

  The words in the puzzle can be entered on a single, or on separate
  lines.  Here is an explicit example:

  \begin{example}
    \begin{wordle}{ANVIL}
      CRANE
      NATAL
      ANVIL
    \end{wordle}
  \end{example}

  As in the online puzzles, the \keyword{wordle} environment
  automatically colours the letter tiles
  \keyword[colback=ForestGreen!80!LimeGreen,colupper=White]{green}
  when the letter is in the same position as the solution and as
  \keyword[colback=Gold!70!DarkGoldenrod,colupper=White]{gold} when
  the letter appears in the solution but is in the wrong position. As
  described below, these colours can be changed to suit your tastes.

  The number of letters in the puzzle is determined by the solution,
  being constrained only by the page width.

  \begin{example}
    \begin{wordle}[rows=4]{orange}
       normal
       orders
       orange
    \end{wordle}
  \end{example}

  The letters in the puzzle can be entered in upper case, lower case, or
  mixed case and, by default, the case of tyhe letters is ignored when
  determining the colours the letter tiles.  There is an array of
  different options that allow you to add extra rows to the puzzle,
  change the colours, fonts, case sensitivity, disable special
  formatting of the \textit{letter tiles} and so on.

  The \keyword|wordle| environment can be customised using the
  following options:
  \begin{quote}\sffamily\color{DarkGreen}
    border,
    correct,
    font,
    letterstyle,
    name,
    placed,
    rows,
    separation,
    shape,
    size,
    strict,
    text,
    tikz,
    unmarked,
    wrong,
    case~sensitive,
    lower case,
    natural case,
    upper case,
  \end{quote}
  These options can be set locally, as the optional argument to a
  \keyword|wordle| environment, by giving a comma separated list
  key-value pairs, or they can be set globally (that is, in the current
  \LaTeX\ group), or using the \keyword{\WordleSetup} command
  \index{\textbackslash WordleSetup}or as package options, via
  \keyword{\usepackage[options]{wordle}}. The order of the options
  does not matter, except that later options take precedence over
  earlier ones, with the exception of \option{tikz} and
  \option{letterstyle} are always applied last.

  The \keyword{wordle} puzzles are drawn using \ctan{TikZ}, with each
  letter in the puzzle sitting in a \ctan{TikZ} node. Many of the
  options control the settings of these nodes. Several of these options
  set the colours used by \keyword{wordle}. Colour names use the
  \keyword{svgnames} names provided by the \ctan{xcolor} package.

  The descriptions of the options below are grouped according to their
  function.

  %%%%%%%%%%%%%%%%%%%%%%%%%%%%%%%%%%%%%%%%%%%%%%%%%%%%%%%%%%%%%%%%%%%%%%%%%%%%%%%%%
  \Section{Options controlling the style of the letter tiles}

  The letter tiles in a \keyword{wordle} environment are typeset insude
  a \ctan{TikZ} \keyword{\node}. The options below control the different
  style settings of these nodes.

  \option{border}[white]

  Sets the border colour of the \keyword{wordle} letter tiles.

  \begin{example}
    \begin{wordle}[border=red]{ANVIL}
      CRANE
      NATAL
      ANVIL
    \end{wordle}
  \end{example}

  \option{correct}[ForestGreen!80!LimeGreen]

  Sets the tile colour of the correct \keyword{wordle} letter tiles,
  which are the letters in the same positions as letters in the solution.

  \begin{example}
    \begin{wordle}[correct=LightSkyBlue]{cellar}
      pounds
      tailer
      cellar
    \end{wordle}
  \end{example}

  \option{font}[\textbackslash bfseries]

  Sets the font used by \keyword{wordle} letters. Any valid sequence of
  \LaTeX\ font commands can be used.

  \begin{example}
    \begin{wordle}[font=\large\sffamily]{ANVIL}
      CRANE
      NATAL
      ANVIL
    \end{wordle}
  \end{example}

  \option{placed}[Gold!70!DarkGoldenrod]

  Sets the tile colour of the \textit{placed} \keyword{wordle} letter tiles,
  which are those letters that appear in the solution but which are
  currently in the wrong position.

  \begin{example}
    \begin{wordle}[placed=Orange]{ANVIL}
      CRANE
      NATAL
      ANVIL
    \end{wordle}
  \end{example}

  \option{separation}[1mm]

  Sets the separation distance between the \keyword{wordle} letter
  tiles.

  \begin{example}
    \begin{wordle}[separation=2mm]{ANVIL}
      CRANE
      NATAL
      ANVIL
    \end{wordle}
  \end{example}

  \option{shape}[rectangle]

  Specifies the \textit{shape} of the \keyword{wordle} letter. The default tile
  \option{shape} is a \textsf{rectangle} (more precisely, a square), but the
  \ctan{TikZ} node shapes of \textsf{circle} and \textsf{coordinate} can
  also be used; see section 17.2 of the \ctan{TikZ} manual.

  \begin{example}
    \begin{wordle}[shape=circle]{LADLE}
       FLAIL LAPEL * LADLE LOVER
    \end{wordle}
  \end{example}

  \option{size}[7.0mm]

  The \option{size} option sets the size of the \keyword{wordle}
  letter tiles.  If you change the size  of the tiles, then you will
  almost certainly want to change font size font as well, which you can
  do using the \option{font} option:

  \begin{example}
    \begin{wordle}[size=10mm, font=\huge\bfseries]{LADLE}
       FLAIL LAPEL * LADLE
    \end{wordle}
  \end{example}

  \option{text}[white]

  Sets the colour of the letters in the \keyword{wordle} puzzle.

  \begin{example}
    \begin{wordle}[text=DarkSlateGrey]{LADLE}
       FLAIL LAPEL LASER
    \end{wordle}
  \end{example}

  \option{wrong}[Gray]

  Sets the tile colour of the \keyword{wordle} letter tiles for letters
  that do nbot appear in the solution.

  \begin{example}
    \begin{wordle}[wrong=red]{LADLE}
       FLAIL LAPEL LASER
    \end{wordle}
  \end{example}

  \option{letterstyle}[--]

  A comma-separated list of \ctan{Tikz} node keys that are appended to
  the style of the \ctan{TikZ} node that contains each of the
  \keyword{wordle} letters.  Most of the options given above can also be
  set using the \option{letterstyle} option.

  With most of the \keyword{wordle} options, the latter options take
  precedence, however, the \option{letterstyle} option is applied last,
  so it overrides any competing options. When using the
  \option{letterstyle} option, you should be careful not to break the
  special formatting of letters provided by the \keyword{wordle}
  environment.\par

  \begin{example}
    \begin{wordle}[letterstyle={rounded corners}]{ANVIL}
      CRANE
      NATAL
      ANVIL
    \end{wordle}
  \end{example}

  %%%%%%%%%%%%%%%%%%%%%%%%%%%%%%%%%%%%%%%%%%%%%%%%%%%%%%%%%%%%%%%%%%%%%%%%%%%%%%%%%

  \Section{Extra rows and columns}

  \option{rows}[0]

  \option{strict}[false]

  By default, the \keyword{wordle} environment does not check to make
  sure that each word in the puzzle has the same length as the solution,
  so missing and extra letters receive no special processing.

  \begin{example}
    \begin{wordle}{CRAZE}
      GRAIL
      TRACK
      CRAMPS
      CRAM
      CRABS
    \end{wordle}
  \end{example}

  The \option{strict} option enforces the word length in the puzzle by
  marking missing letters as incorrect and striking out extra letters.

  \begin{example}
    \begin{wordle}[strict]{CRAZE}
      GRAIL
      TRACK
      CRAMPS
      CRAM
      CRABS
    \end{wordle}
  \end{example}

  \option{unmarked}[white]
  Sets the tile colour of the \textit{unmarked } \keyword{wordle} letter
  tiles, which are neither correct or placed.

  %%%%%%%%%%%%%%%%%%%%%%%%%%%%%%%%%%%%%%%%%%%%%%%%%%%%%%%%%%%%%%%%%%%%%%%%%%%%%%%%%

  \Section{Case sensitivity}
  \index{case~sensitive}
  \index{case~sensitive!lower case}
  \index{case~sensitive!natural case}
  \index{case~sensitive!upper case}

  By default, the letters are printed in upper and lower case, exactly
  as typed, and the letter comparisons are not case sensitive.

  \option{case~sensitive}[false]

  By default the \keyword{wordle} puzzles are not case sensitive, but
  you can make them case sensitive using this option. Note that
  \keyword{case sensitive} is the same as \keyword{case sensitive=true}
  (and the default option is the same as
  \keyword{case sensitive=false}).

  \begin{example}
    \begin{wordle}{youth}
      North
      South
      Youth
    \end{wordle}
  \end{example}

  Here is a more (case) sensitive version of this puzzle.

  \begin{example}
    \begin{wordle}[case sensitive=true]{youth}
      North
      South
      Youth
    \end{wordle}
  \end{example}

  As this puzzle is case sensitive, the \keyword{Y} on the last line
  is not correct because it does not match the \keyword{y} in the
  solution.

  The \option{upper case} and \option{lower case} options force all
  letters in the puzzle to be in upper or upper case, respectively. Both
  of these options disable the case sensitivity of the tests:

  \begin{example}
    \begin{wordle}[lower case]{youth}
      North
      South
      Youth
    \end{wordle}
  \end{example}

  \begin{example}
    \begin{wordle}[upper case]{youth}
      North
      South
      Youth
    \end{wordle}
  \end{example}

  To balance these two options, the \option{natural case} option makes
  the \keyword{wordle} environment respect the case of the letters as
  typed into the environment.

  \begin{example}
    \begin{wordle}[natural case]{youth}
      North
      South
      Youth
    \end{wordle}
  \end{example}

  %%%%%%%%%%%%%%%%%%%%%%%%%%%%%%%%%%%%%%%%%%%%%%%%%%%%%%%%%%%%%%%%%%%%%%%%%%%%%%%%%

  \Section{Adding \ctan{TikZ} code}

  The \keyword{wordle} environment is a fancy wrapper around a
  \keyword{tikzpicture} environment, with some fancy code under the hood
  to determine the colour of the letter tiles. The second optional
  argument to the \keyword{wordle} environment can be used to add
  further \ctan{TikZ} commands to the puzzle.

  \option{name}[W]

  \option{tikz}[]

  \begin{example}
    \begin{wordle}[border=red, rows=4, case sensitive]{Ladle}
      [{
        \draw[blue,ultra thick](W1-2.north east)--(W3-1.center);
      }]
      Flail lapel * Ladle
    \end{wordle}
  \end{example}

  %%%%%%%%%%%%%%%%%%%%%%%%%%%%%%%%%%%%%%%%%%%%%%%%%%%%%%%%%%%%%%%%%%%%%%%%%%%%%%%%%

  \Section{The package code}

  The \keyword{wordle} package is a \LaTeX3 package. There are many
  comments in the code explaining how the code works. The automatic
  colouring of the letter tiles is a fun exercise in \LaTeX3
  programming, which was motivated by a
  \href{https://tex.stackexchange.com/questions/659860/wordle-like-colored-letter-boxes-in-latex/660056#660056}{tex.stackexchange.com} post.

  %%%%%%%%%%%%%%%%%%%%%%%%%%%%%%%%%%%%%%%%%%%%%%%%%%%%%%%%%%%%%%%%%%%%%%%%%%%%%%%%%
  \printindex

\end{document}
